\documentclass[12pt, a4paper]{article}
\usepackage{hyperref}
\usepackage{graphicx}
\usepackage{enumitem}

\pagenumbering{gobble}

\newcommand{\bs}{\textbackslash}

\title{Fortran pipeline FiWi Würzburg - Manual}
\author{Lukas K. Schumann, \\\small{lukas\_kilian.schumann@stud-mail.uni-wuerzburg.de}}
\date{Stand: 03.2022}

\begin{document}

\maketitle

\bigskip\noindent

\begin{enumerate}[leftmargin=-1em]
    \item bla
\end{enumerate}



\huge{Installation}

- Generiere ssh-keys (wurde vermutlich schon erledit, und muss nicht mehr gemacht werden)
Im Skript muss der hostname auf den eigenen Benutzeraccount gesetzt werden.
Anschließend kann das heruntergeladene script mit --install aufgerufen werden

Einmal terminal öffnen, und dort ssh user@10.106.242.102 eingeben.

Hier muss einmalig mit Ja bzw. Yes bestätigt werden, dass der Server in die Liste der bekannten Server aufgenommen wird.


\huge{Nutzung}
Zur nutzung das programm in den projektordner kopieren und mit doppelklick starten.
Es folgt eine auflistung aller im aktuellen Verzeichnis befindenden .f90 Dateien.
Die zu startende kann einfach durch drücken der entsprechenden Zahl bestätigt werden.
Weiter benötigte Dateien, wie zb für parameter werden automatisch mit übermittelt, sofern sie sich im selben Verzeichnis befinden.


\huge{Problemlösung}


\end{document}

